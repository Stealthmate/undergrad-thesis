自転車共有問題 (BS) とは,直線上にある点 0 に置かれた人及び自転車を点 1 まで運送することを考える問題であり,自転車の台数が人の人数以下である時,自転車も人も含めていかに早く到着させるかを目的としている.これを満たす最適な移動スケジュールは多項式時間で計算可能であることが知られている.本論文では条件を緩和し,一部の人が出発点と到着点の間の任意な点から出発する問題 (FSABS) を考え,その解法を求める.具体的には, FSABS に対する下界を定めてから, BS を部分問題として解くことで FSABS に対しても多項式時間で最適解が求められることを示す.

本論文の構成は以下の通りである.\chapref{introduction} では自転車共有問題の背景及び,本研究で考えた拡張である FSABS について概ねの動機とアプローチを紹介する.\chapref{bs} では \textcite{czyzowicz} による自転車共有問題の定義を与え,最適解の下界及びそれを達成する実行可能解すなわち最適解を求める多項式時間アルゴリズムについて紹介する.\chapref{fsabs} では FSABS を定義し,その解に関わるいくつかの性質を述べてから, BS と同様な方式で下界を定める.その次に,下界を達成する実行可能解すなわち最適解を求めるアルゴリズム \textsc{Solve-FSABS} を提案し, \textsc{Solve-FSABS} が実際に最適解を出力すること及び \textsc{Solve-FSABS} の時間計算量が人数の多項式で与えれることを示す.そして最後に \chapref{conclusion} では本研究に基づいた予想及び BS に対するさらなる拡張について簡潔に述べる.