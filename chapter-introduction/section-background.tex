現代社会では「ロボット」という概念は良く知られている.ロボットとはコンピューターにより制御される機械であり,人間に代わって何かの作業を自動的にかつ自律的に行うためのものである~\cite{dict:cambridge:robot,dict:mw:robot,dict:daijirin:robot}.しかし,近年では「コボット」 (cobot) というものの展開への関心が高まっている~\cite{4154820,6386300}.コボットとは collaborative robot の略称として,ロボットとは対照に自律的ではなく,人間により制御されるが,一緒に作業する人間の能力を補助・強化することを目的としたものである~\cite{patent:cobots}.

「コボット」という名称は本来ならば高度な作業を行う機械を意味するが,単純な一例として多くの人の身近にある「自転車」をコボットとして扱うことができる.自転車は自律行動ができないが,人間に制御されることにより人間の作業を補助する (移動速度を上げる).実際には輸送に関わる問題を考えるときにこのようなモデルは有用である.実際にこのモデルを用いた問題として \textcite{czyzowicz} により提案された Bike Sharing 問題 (自転車共有問題もしくは単に BS) がある.自転車共有問題とは,直線上にある点 0 に置かれた人及び自転車を点 1 まで運送することを考える問題であり,自転車の台数が人の人数以下である時,自転車も人も含めていかに早く到着させるかを目的としている.この問題の解法は同じく Czyzowicz らにより提案され,後程説明するスケジュールというものを解としたときに,最適な解が人数に対する多項式時間で求まる.

しかし,全ての人及び自転車が同じところから出発するという状況は現実的に珍しいと考えられる.そこで本論文では条件を緩和し,一部の人が出発点と到着点の間の任意な点から出発する問題を考える.ただし,人が後ろに移動することを許さないため,点 0 においては少なくとも自転車と同じ数の人間がいなければならないことに注意する.この問題を Forward Scattered Agents Bike Sharing 問題 (前方に人がいることを許した自転車共有問題もしくは単に FSABS) と名付ける.本論文では FSABS の解法アルゴリズム \SolveFSABS{} を求め,それが最適解を出力すること及び人数に対する多項式時間で終了することを示す.