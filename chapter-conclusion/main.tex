本論文では自転車共有問題 (BS) について紹介し,その拡張として考えられる前方に人がいることを許した自転車共有問題 (FSABS) を定義した.また, FSABS の解法としてアルゴリズム \textsc{Solve-FSABS} を与え,部分問題として BS を解くことによって FSABS が多項式時間で解けることを示した.

今回紹介したアルゴリズムは計算途中の状態として \condref{riders-order} の下で出発点より後ろにライダーがいること,つまり FSABS よりさらに拡張された入力を許している.筆者は前後関係なく任意の位置に任意の人及び自転車を置くことを許した問題でも \textsc{Solve-FSABS} と同様な方法で解くことができると予想している.具体的には,同じく再帰的な形でアルゴリズムを組み, 1 つのグループに注目するのではなく,単独移動も含めてあらゆるグループの中で一番早くできる 2 組を合流させ, \text{Solve-FSABS} と同時に他のグループをそれぞれそのまま移動させるという方法が最適だと考えられる.

他方で後退を許し,前方の他に後方に人や自転車を置いた問題も考えることができる.その場合,移動距離の上界がなくなるため,最適な到着時間を直接計算することでさえ難しいと考えられる.その一方で後退を許した方が現実に応用できるので,より価値がある問題だと考えられる.